\subsubsection{Def. Reduktioner}

Det bringer os så til en grundlæggende metode i computationel
kompleksitetsteori, nemlig reduktioner. Først og fremmest den formelle
definition.

Givet to sprog $L_1$ og $L_2$, en polynomiel reduktion $r$ af $L_1$ til $L_2$ er
en polynomial time computable map for hvilken der gælder:

\begin{align*}
 \forall x : x \in L_1 \text{ hviss. } r(x) \in L_2
\end{align*}

Dette skrives som $L_1 \leq L_2$, hvor man læser det som at $L_1$ reduceres
til $L_2$. Intuitivt betyder reduktion blot, at vi kan oversætte enhver given
instans af $L_1$ til en anden instans af $L_2$. Vi siger desuden, at $L_2$ er et
mere generelt sprog end $L_1$ og kan derfor ses, som værende mere sandsynlig
til ikke at være i $P$.


Herudover har reduktioner desuden to nyttige egenskaber vi skal bruge senere.

\paragraph{Proposition 3: Hvis $L_1 \leq L_2$ og $L_2 \leq L_3$, så gælder der
$L_1 \leq L_3$}
~\\
~\\
Denne proposition underbygger, at reduktioner er transitive. Vi beviser den
således:

\begin{proof}
 Vi har polynomial time computable maps $r_1()$ og $r_2()$, hvor følgende ting
gælder:

\begin{itemize}
 \item For ethvert $x$ gælder der, at $x \in L_1$ hvis og kun hvis $r_1(x) \in
L_2$.
 \item For ethvert $y$ gælder der, at $y \in L_2$ hvis og kun hvis $r_2(y) \in
L_3$.
\end{itemize}

Således har vi for alle $x$, at $x \in L_1$ hvis og kun hvis $r_2(r_1(x))
\in L_3$. Og siden vi blot har brugt to polynomial time computable maps efter
hinanden (hvormed det hele kan ses som en polynomial time computable map), så
har vi $L_1 \leq L_3$.
\end{proof}

\paragraph{Proposition 4: Hvis $L_1 \leq L_2$ og $L_2 \in P$, så er $L_1 \in P$.}

Proposition 4 siger intuitivt, at $P$ er lukket nedad under reduktion. 

