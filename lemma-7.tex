\subsubsection{Lemma 7}

For at kunne bevise et givent sprog er NP-hard, så bruger vi polynomielle
reduktioner fra kendte NP-hard problemer, fremfor at forsøge et direkte bevis.
For at kunne lave disse reduktioner, så har vi brug for følgende lemma.\\
~\\
\textbf{Lemma 7:} Hvis $L_1$ er NP-hard og $L_1 \leq L_2$ så er $L_2$ også
NP-hard.

\begin{proof}
 Siden $L_1$ er NP-hard, så ved vi alle sprog $L$ i NP kan reduceres til
det ($\forall L \in NP: L \leq L_1$). Så når $L_1 \leq L_2$, så kan vi
bruge transitivitetsreglen (Proposition 3) og konkludere at ethvert sprog $L$
reducerer til $L_2$ ($\forall L \in NP: L \leq L_1 \leq L_2$), hvormed $L_2$
altså er NP-hard.
\end{proof}

