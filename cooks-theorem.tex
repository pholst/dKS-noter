\subsubsection{Cook's Theorem}

Det bringer os så videre til hovedemnet her, nemlig Cook's Theorem. Vi vil jo
gerne kunne vise at diverse problemer er NP-Complete, men for at gøre det vha.
reduktioner, så skal vi først have et NP-Complete problem at starte ud fra.

Takket være Stephen Cook fik vi i 1972 netop lige det, da han viste at
SATISFIABILITY PROBLEMET, forkortet SAT, var NP-hard. Herefter kunne mange andre
så bruge SAT til at reducere til andre problemer, for at vise at disse nye
problemer var NP-hard. Cook beviste oprindeligt SAT ved at vise alle problemer i
NP reducerede til SAT, hvilket er en noget kompliceret affære. Derfor vil vi i
stedet vise et relateret problem CIRCUIT SAT er NP-hard og så reducerer dette
til SAT.

Lad os derfor først og fremmest se på definitionerne på disse to problemer og
derefter gå i gang med de egentlige to dele af beviset:

\begin{itemize} \item CIRCUIT SAT $\in$ NPC (Theorem 11) \item CIRCUIT SAT
$\leq$ SAT (Proposition 12) \end{itemize}

\paragraph{Def. SAT} ~\\ ~\\ Givet en CNF formel, er der en tildeling af
sandt/falsk til variablerne således hele udtrykket evaluerer til sandt?\\ ~\\
\textit{Note: SAT $\in$ NP, siden vi kan evaluere en given boolsk funktion på
en bitvektor i polynomiel tid og verificere hvorvidt vi har en tilfredsstillende
tildeling. At finde en tilfredsstillende tildeling direkte kræver dog
exhaustive search i eksponentielt mange muligheder.}

\paragraph{Def. CIRCUIT SAT} ~\\ ~\\ Givet et boolsk kredsløb $C$, er der
en inputvektor $x \in \left\lbrace 0,1 \right\rbrace^n$ således at $C(x) =
1$? ~\\ \textit{Note: CIRCUIT SAT $\in$ NP, siden vi kan evaluere et givent
kredsløb på en inputvektor i polynomiel tid og verificere hvorvidt vi har en
tilfredsstillende tildeling. At finde en tilfredsstillende tildeling direkte
kræver derimod exhaustive search eksponentielt mange muligheder.}

